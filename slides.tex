\documentclass[aspectratio=169]{beamer}



\mode<presentation>
{
 \usetheme[reversetitle,notitle,noauthor]{Wien} 
%  \usetheme[noauthor]{Wien} 
}       
  
\usepackage{url}
\usepackage{graphicx}
\graphicspath{{./}{./Figures/}}  

\usepackage{appendixnumberbeamer} 

% To avoid a warning from the hyperref package:
\pdfstringdefDisableCommands{%
  \def\translate{}%
}

% To make sure, that the footnote is placed above and outside the
% footline (but it only works for one footnote per frame):
% 
% \addtobeamertemplate{footnote}{}{\vspace{4ex}}

%%%%%%%%%%%%%%%%%%%%%%%%%%%%%%%%%%%%%%%%%%%%%%%%%%%%%%%%%%%%%%%%%%%%%%%%%%%%% 
%%%%%%%%%%%%%%%%%%%%%%%%%%%%%%%%%%%%%%%%%%%%%%%%%%%%%%%%%%%%%%%%%%%%%%%%%%%%%
\title[ICT Presentations]{10 HEURISTICAS DE USABILDAD DE NIELSEN Y MOLICH}


\subtitle{Interaccion Humano Computador}

\author[A. Jantsch]{Patrik Renee Quenta Nina\\
Frank Lenny Ccapa Usca\\
Alfred Marvin Casanova Vargaya\\
Yhonatan Smith Ylaccaña Cordova
}
   
\date{Septiembre 14, 2020}


\begin{document}

\begin{frame}
  \titlepage
\end{frame}      

%%%%%%%%%%%%%%%%%%%%%%%%%%%%%%%%%%%%%%%%%%%%%%%%%%%%%%%%%%%%%%%%%%%%%%%%%%%%% 
%%%%%%%%%%%%%%%%%%%%%%%%%%%%%%%%%%%%%%%%%%%%%%%%%%%%%%%%%%%%%%%%%%%%%%%%%%%%%
%%%%%%%%%%%%%%%%%%%%%%%%%%%%%%%%%%%%%%%%%%%%%%%%%%%%%%%%%%%%%%%%%%%%%%%%%%%%%

\begin{frame}{INTRODUCCION}
  
\begin{itemize}
\item Jakob Nielsen es un consultor de usabilidad web, tiene un doctorado en la interacción persona-computadora HCI  en la Universidad Técnica de Dinamarca en Copenhague,  partner de Nielsen Norman Group, y Rolf Molich.
\item Rolf Molich es un consultor Danés, padre de la usabilidad y que co invento junto a Nielsen el método de evaluación heurística.
\end{itemize}
\end{frame}


%%%%%%%%%%%%%%%%%%%%%%%%%%%%%%%%%%%%%%%%%%%%%%%%%%%%%%%
\begin{frame}{10 HEURISTICAS DE USABILDAD DE NIELSEN Y MOLICH}

  \begin{block}{}
    Nielsen Y Molich establecieron una lista de 10 pautas a seguir de diseño de interfaz de usuario en el año 1990.
  \end{block}
  \vspace{3ex}
  \begin{block}{}
    Las 10 heurísticas de usabilidad tienen como objetivo mejorar la experiencia de usuario dentro de una interfaz humano-computadora o humano-ordenador si queremos llamarle así. Usabilidad es la facilidad con el usuario interactúa con una herramienta con el fin de alcanzar un objetivo concreto.
  \end{block}
\end{frame} 

%%%%%%%%%%%%%%%%%%%%%%%%%%%%%%%%%%%%%%%%%%%%%%%%%%%%%%%
\begin{frame}{Visibilidad del estado del sistema}
  
  \begin{block}{Visibilidad del estado del sistema.}
  \begin{itemize}    
    \item El usuario debe estar informado de las operaciones del sistema, estás operaciones deben ser altamente visible y que se muestre en pantalla dentro de un periodo de tiempo razonable de manera que para el usuario sea fácil de entender.

  \end{itemize}
  \end{block}

\end{frame}

%%%%%%%%%%%%%%%%%%%%%%%%%%%%%%%%%%%%%%%%%%%%%%%%%%%%%%%
\begin{frame}{Relación entre el sistema y el mundo real}
  
  \begin{block}{Relación entre el sistema y el mundo real.}
  \begin{itemize}    
    \item Se debe reflejar el lenguaje y los conceptos que los usuarios encontrarían en el mundo real en función del usuario objetivo del estudio. La información debe ser presentada en orden lógico y la combinación de las expectativas del usuario derivadas de su experiencia en el mundo real reducirá la tensión cognitiva y facilitará el uso del sistema.

  \end{itemize}
  \end{block}

\end{frame}

  %%%%%%%%%%%%%%%%%%%%%%%%%%%%%%%%%%%%%%%%%%%%%%%%%%%%%%%
  
\begin{frame}{Control y libertad del usuario}
  
  \begin{block}{Control y libertad del usuario.}
  \begin{itemize}    
    \item El usuario debe tener la opción de hacer pasos hacia atrás en cualquier proceso y que sean posibles incluyendo rehacer y deshacer acciones anteriores.

  \end{itemize}
  \end{block}

\end{frame}

%%%%%%%%%%%%%%%%%%%%%%%%%%%%%%%%%%%%%%%%%%%%%%%%%%%%%%%
\begin{frame}{Control y libertad del usuario}
  
  \begin{block}{Control y libertad del usuario.}
  \begin{itemize}    
    \item El usuario debe tener la opción de hacer pasos hacia atrás en cualquier proceso y que sean posibles incluyendo rehacer y deshacer acciones anteriores.

  \end{itemize}
  \end{block}

\end{frame}

%%%%%%%%%%%%%%%%%%%%%%%%%%%%%%%%%%%%%%%%%%%%%%%%%%%%%%%
\begin{frame}{Consistencia y normas}
  
  \begin{block}{Consistencia y normas.}
  \begin{itemize}    
    \item EL diseñador de la interfaz debe asegurar que los elementos gráficos y su terminología se mantenga en todas las pantallas o plataformas similares. Los iconos de una categoría deben representar el mismo concepto en todas las pantallas.


  \end{itemize}
  \end{block}

\end{frame}
%%%%%%%%%%%%%%%%%%%%%%%%%%%%%%%%%%%%%%%%%%%%%%%%%%%%%%%
\begin{frame}{Prevención de errores}
  
  \begin{block}{Prevención de errores.}
  \begin{itemize}    
    \item Diseñar sistemas para que los errores potenciales se mantengan al mínimo. El usuario no debe detectar y solucionar problemas sino al contrario, el sistema le debe ofrecer posibles soluciones.

  \end{itemize}
  \end{block}

\end{frame}
%%%%%%%%%%%%%%%%%%%%%%%%%%%%%%%%%%%%%%%%%%%%%%%%%%%%%%%
\begin{frame}{Reconocer en lugar de recordar}
  
  \begin{block}{Reconocer en lugar de recordar.}
  \begin{itemize}    
    \item El diseñador debe crear interfaces que hagan que los objetos, acciones, opciones y direcciones sean visibles y fácilmente reconocibles para el usuario.
    
    \item Se debe minimizar los datos y mostrar solo información relevante dentro de una pantalla mientras el usuario explora la interfaz. Las personas solo podemos mantener alrededor de 5 elementos en la memoria al mismo tiempo y en corto plazo. Debido a estas limitaciones el diseñador debe simplemente emplear el reconocimientos en lugar de hacer recordar información.
    
    \item Reconocer es más fácil que recordar.

  \end{itemize}
  \end{block}

\end{frame}

%%%%%%%%%%%%%%%%%%%%%%%%%%%%%%%%%%%%%%%%%%%%%%%%%%%%%%%
\begin{frame}{Flexibilidad y eficiencia de uso}
  
  \begin{block}{Flexibilidad y eficiencia de uso.}
  \begin{itemize}    
    \item Al mayor uso de un sistema viene la demanda de menos interacciones que permitan una navegación más rápida. Los usuarios deben poder personalizar o adaptar la interfaz para que se adapte a sus necesidades, de modo que se pueda lograr acciones frecuentes a través de maneras mas convenientes.

  \end{itemize}
  \end{block}

\end{frame}

%%%%%%%%%%%%%%%%%%%%%%%%%%%%%%%%%%%%%%%%%
\begin{frame}{Diseño estético y minimalista}
  
  \begin{block}{Diseño estético y minimalista.}
  \begin{itemize}    
    \item No se debe mostrar información innecesaria lo más importante es que se debe mostrar solo los componentes necesarios para desarrollar tareas, de manera claramente visibles y que sean inequívocos para ayudar al usuario a  navegar de un sitio a otro. La información innecesaria podría distraer al usuario y disminuir la atención sobre la información relevante.

  \end{itemize}
  \end{block}

\end{frame}

%%%%%%%%%%%%%%%%%%%%%%%%%%%%%%%%%%%%%%%%%
\begin{frame}{Ayudar a los usuarios a reconocer, diagnosticar y corregir errores}
  
  \begin{block}{Ayudar a los usuarios a reconocer, diagnosticar y corregir errores.}
  \begin{itemize}    
    \item Los errores deben expresarse en un lenguaje sencillo por sobre la terminología técnica según el ususario al que este dirigido está acción.

  \end{itemize}
  \end{block}

\end{frame}

%%%%%%%%%%%%%%%%%%%%%%%%%%%%%%%%%%%%%%%%%

%%%%%%%%%%%%%%%%%%%%%%%%%%%%%%%%%%%%%%%%%


%%%%%%%%%%%%%%%%%%%%%%%%%%%%%%%%%%%%%%%%%
\begin{frame}{}
  \centering\Huge
  GRACIAS
\end{frame}

%%%%%%%%%%%%%%%%%%%%%%%%%%%%%%%%%%%%%%%%%%%%%%%%%%%%%%%%%%%%%%%%%%%%%%%%%%%%% 
%%%%%%%%%%%%%%%%%%%%%%%%%%%%%%%%%%%%%%%%%%%%%%%%%%%%%%%%%%%%%%%%%%%%%%%%%%%%%

\end{document}


%%% Local Variables:
%%% mode: latex
%%% TeX-master: t
%%% End:
